\noindent
\begin{longtable}{|>{\raggedleft\hsize=.7\hsize\bfseries}X|
    >{\arraybackslash\hsize=1.3\hsize}X|} \hline
\multicolumn{1}{|l|}{\textbf{Use Case}} &  Bekijk trajectmap\\ \hline
Primaire actor & Gebruiker\\ \hline
Stakeholders & Operator \\ \hline
Preconditie &  De data van de trajectmap is beschikbaar.\\ \hline
Postconditie &  De trajectmap wordt getoond waarop de routes worden weergeven op een kaart met indicatie van hun reistijd.\\ \hline
\multicolumn{1}{|l|}{\textbf{Normaal verloop}} & \\ \hline
1. & De gebruiker wenst de trajectmap op te vragen.\\ \hline
2. & Het systeem haalt de recentste gegevens op om de trajectmap te vormen.\\ \hline
3. & Het systeem toont de trajectmap.\\ \hline
\multicolumn{1}{|l|}{\textbf{Alternatief verloop}} & \\ \hline
3A. & Het systeem kan de trajectmap niet tonen. \\ \hline
3A1. & Het systeem geeft een correcte boodschap. \\ \hline
3A2. & Terug naar stap 1. \\ \hline
\multicolumn{1}{|l|}{\textbf{Domeinspecifieke regels}} & \\ \hline
\multicolumn{1}{|l|}{\textbf{Op te klaren punten}} & De filters worden pas achteraf toegepast. Daarom zit dit niet in deze use case.\\ \hline
\caption{UC: Trajectmap bekijken \label{uc:mapbekijken}}
\end{longtable}