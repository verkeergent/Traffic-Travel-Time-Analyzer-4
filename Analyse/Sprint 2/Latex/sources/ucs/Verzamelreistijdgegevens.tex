\noindent
\begin{longtable}{|>{\raggedleft\hsize=.7\hsize\bfseries}X|
    >{\arraybackslash\hsize=1.3\hsize}X|} \hline
\multicolumn{1}{|l|}{\textbf{Use Case}} &  Verzamel reistijdgegevens\\ \hline
Primaire actor & Polling Service \\ \hline
Stakeholders & \\ \hline
Preconditie &  Er zijn minimaal 5 minuten verstreken sinds de laatste aanvraag van reisdata gegevens. 

Er zijn routes beschikbaar in de database.\\  \hline
Postconditie &  De reistijden van elke route zijn voor elke provider geüpdatet. \\ \hline
\multicolumn{1}{|l|}{\textbf{Normaal verloop}} & \\ \hline
1. & De polling service wenst recente reistijd gegevens op te halen. \\ \hline
2. & Het systeem haalt alle beschikbare routes op. \\ \hline
3. & Het systeem scrapet de reistijden en vertragingen van hun respectievelijke websites.\\ \hline
5. & De polling service verwerkt de verzamelde gegevens en slaat deze op in de databank.\\ \hline
6. & De polling service vraagt (parallel) aan alle per-route providers de recenste reistijden per route.\\ \hline
7. & Het systeem scrapet de reistijden en vertragingen van hun respectievelijke website. \\ \hline
8. & De polling service wacht tot alle resultaten bekend zijn, verwerkt de gegevens en slaat deze op in de databank. \\ \hline
\multicolumn{1}{|l|}{\textbf{Alternatief verloop}} & \\ \hline
& Voor TomTom, Google en Here maps:\\ \hline
6A. & De reistijd gegevens kunnen niet opgevraagd worden via scraping. \\ \hline
6A1. & De provider vraagt de reistijd gegevens op aan de API van de provider met de opgeslagen API keys\\ \hline
6A2. & Ga naar stap 7\\ \hline
\multicolumn{1}{|l|}{\textbf{Domeinspecifieke regels}} & \\ \hline
\multicolumn{1}{|l|}{\textbf{Op te klaren punten}} & \\ \hline
\caption{UC: Verzamel reistijdgegevens \label{uc:gegevensverzamelen}}
\end{longtable}
%