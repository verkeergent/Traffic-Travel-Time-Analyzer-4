Wanneer er sprake is van \textbf{\textit{``providers''}} dan wordt hiermee de verzameling van geïmplementeerde providers bedoelt. Deze bestaan momenteel uit TomTom, Coyote, HereMaps, Google Maps, Bing Maps, Be-mobile en ViaMichelin en Waze. Indien er een providerafhankelijke beslissing werd genomen dan zal dit gespecificeerd worden.

Een operator is altijd een gebruiker. De verschillen zijn weggewerkt en in gebruik van de applicatie zijn beide dezelfde. Uit analytisch standpunt is echter een operator degene die wijzigingen kan aanbrengen aan het systeem (zoals bvb. het wijzigen van een route), terwijl een gebruiker enkel observeert (en data afhaalt). 

\textbf{DR Route-informatie}: Een route heeft een naam, een afstand, een normale reistijd, een huidige reistijd en bijgevolg een vertraging. Deze laatste 3 kunnen afwijken per provider.

\textbf{DR Providers}: Een provider heeft een naam en providerspecifieke eigenschappen.

\textbf{DR Detail Filters}: Op de detailpagina kunnen volgende filters ingesteld worden: startdatum (en tijdstip), einddatum (en tijdstip).

\textbf{DR Vergelijk Filters}: Op de vergelijkroutes-pagina kunnen volgende filters ingesteld worden: twee routes, een startdatum (en tijdstip), een einddatum (en tijdstip) en providers.

\textbf{DR Dashboards}: Op de homepagina bevinden zich zogenaamde ``dashboards''. Dit zijn de verschillende ``panels'' of aspecten op deze pagina. Momenteel is dit een samenvatting van de logs, een overzicht van de POI's, een minikaart met de huidige status, het weer en een overzicht van de laatste tweets van VerkeerGentB.
